The Heavy Photon Search experiment successfully took data during the spring 2015 and 2016 engineering runs. The detectors and beam performed well and consistently with that which was proposed and simulated. This thesis presents details of the simulation and calibration of the ECal and its performance in both runs. The first results of the unblinded, full 2015 vertex analysis using data from multiple SVT layer combinations and different SVT settings is presented. \\
\indent It has been shown that the proposal vertex anticipated exclusion made assumptions about the detector acceptance that are not consistent with Monte Carlo and experimental data. No exclusion is obtained from the 2015 run, but the procedure for searching for displaced vertices and the acceptance-related effects have been fully developed and presented here. The procedure for combining the data sets from different SVT layer combinations has been discussed. L1L2 and L2L2 contributions in the 0.5~mm data have too much background to have a $zCut$ low enough for these data to contribute to the overall reach. The L1L2 and L2L2 contributions in the 1.5~mm data have less excess high $z$ backgrounds, but the statistics of the data is too low to contribute to the reach. Additionally, there are excess background events that contaminate the vertex analysis and will require further study after the improved implementation of the vertex reconstruction code in the near future. \\
\indent HPS has a promising future with more beam time scheduled and planned upgrades that will improve the vertex reach. The addition of a hodoscope in front of the positron side of the ECal will improve the trigger by eliminating triggers from photons from Wide Angle Bremsstrahlung and recovering events where the electron are lost in the ECal hole. A Layer 0 at 5~cm downstream of the target will improve the vertex resolution of the SVT by enabling tighter $zCut$s. The excess backgrounds will continue to be studied from both runs. 