The ECal was used to trigger events and measure particle energy and timing. The ECal is a homogeneous calorimeter comprised of 442 trapezoidal PbWO$_4$ scintillating crystals, each read out by a large area avalanche photodiode (APD) attached to the back of each crystal~\cite{Balossino201789}. The crystals are re-purposed from the former CLAS IC detector and have been upgraded with larger avalanche photodiodes. Each crystal is trapezoidal in shape and 16~cm long with the front and back faces 1.3$\times$1.3~mm$^2$ and 1.6$\times$1.6~mm$^2$, respectively. The calorimeter layout is shown in Fig.~\ref{Figure:ecalface}. 

\begin{figure}[h]
  \centering
      \includegraphics[width=0.85\textwidth]{pics/experiment/ecalface.png}
  \caption[Drawing of the ECal assembly face]{Drawing of the ECal assembly face, looking downstream along the beam direction. The ECal is assembled in two vertical halves and has a gap between allowing for the electron and photon beams. Beam electrons resulting from energy losses in the target are deflected in the dipole field toward the beam right and result in the "sheet of flame". The ECal includes a cut out in order to avoid the high rates that result from detecting these particles.}
  \label{Figure:ecalface}
\end{figure}

\begin{figure}[h]
  \centering
      \includegraphics[width=0.85\textwidth]{pics/experiment/crystal.png}
  \caption[Single ECal module]{Drawing of an ECal crystal in readout configuration.}
  \label{Figure:crystal}
\end{figure}

\begin{figure}[h]
  \centering
      \includegraphics[width=0.5\textwidth]{pics/experiment/ecalAssembly1.png}
  \caption[Photograph of ECal crystals during assembly]{Photograph taken during assembly of the ECal from above. The preamplifiers attached to each crystal are shown on the left. As single layer of wrapped crystals are shown in their tray and the LEDs attached to each crystal are shown on the right.}
  \label{Figure:ecalAssembly1}
\end{figure}

Each crystal is wrapped in a VM2002 reflecting foil in order to increase light collection. The original APDs used by the the IC had a surface area of 5$\times$5~mm$^2$, but these were upgraded for HPS running by replacing each original APD with a large area APD (model S8664-1010) of surface area 10$\times$10~mm$^2$. The upgraded APDs collect four times more light than the old APDs. The larger signals require less electronic amplification of the signal and improve the signal-to noise-ratio. This allows a lower energy threshold for module readout and improves the energy resolution. \\
\indent The ECal is constructed as two separate vertical halves in order to avoid the 15~mrad vertical zone of excessive electromagnetic background along the beam line. The crystals in each half are arranged in five layers of 46 crystals. The layer of crystals closest to the beam in each half has nine crystals removed to allow for the passing of the unscattered electron beam. The two halves of the ECal are each at 2~cm from the  horizontal electron beam plane.\\
\indent As a particle enters the ECal, it initiates an electromagnetic shower by either bremsstrahlung or pair production. The secondary particles then produce more particles through bremsstrahlung and photon pair production giving rise to a cascade of particles, decreasing in energy. After the electron energy is too low to yield further particles, the remaining energy is deposited through ionization and excitation. The resulting scintillation photons hit the APD that converts the collected photons into an electronic signal via the photoelectric effect. Each APD is attached by a twisted pair connector to a preamplifier which converts the signal current to voltage and has low input impedence and noise.  



The gain of the APDs and the scintillation of the crystals in the ECal are temperature-dependent. An Anova A-40 external chiller operating at 17$\degree$C pumps cool water through copper cooling pipes that run along the inside of the ECal at the top, bottom, front and back face of the crystal structure. The internal temperature of the ECal was monitored using sixteen thermocouples located at various locations within the ECal structure. The thermocouples are readout using an Omega D5000 series transmitters. Both devices are connected through RS-232 serial communications for external monitoring and alarms should the temperature change significantly.\\
\indent Low voltage is supplied to the preamplifiers via an Agilent 6221 operating at 5~V and approximately 4.1~A when all preamplifiers are connected. The high voltage to each of the 52 APD groups is supplied by the CAEN A1520P modules in the SY4527 mainframe. Both voltage suppliers are monitored and accessible for remote operations.  
