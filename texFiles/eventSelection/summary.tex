The events relevant to the vertex analysis were triggered using the pairs-1 HPS trigger cuts. The data is loosely selected as pairs that have a cluster in the top and bottom of the Ecal. A radiative cut, a cut on the measured sum of the energy of the two reconstructed particles in order to selected A' events of interest, is chosen to greater than 80$\%$ of the beam energy. Tracks are reconstructed using various methods and a Generalized Broken Lines (GBL) track re-fit is performed using a minimum of five hits in a track. A vertex is constructed using the "UnconstrainedV0Candidates" collection. An unconstrained vertex studies the closest approach of an $e+e-$ track pair to construct a vertex. While the beamspot constrained vertex collection was not used for measuring vertex position of a pair, the $\chi^{2}$ of the beamspot constrained vertex fit described the quality of how closely the two tracks pass eachother in addition to how well the total momentum projects back to the beamspot position at the target. 
\indent The beamspot constrained collection is not used as the primary collection for studying the vertex position because the beamspot position at the target is not precisely known. If using the z vertex position from a beamspot constrained fit, the size of the beamspot weights the significance of the beamspot as a factor in the vertex fit. A bad beamspot position can systematically pull the measured vertex position. For genuine A' displaced vertices, the z vertex position  is relatively unchanged when using a beamspot vs an unconstrained vertex fit. For identifying backgrounds, prompt vertices that are pushed to large z through measurement error, the beamspot constraint can arbitrarily bias the location of the scattered vertex. In order to avoid bias to the reconstructed z vertex position, the unconstrained fit is used.  \\
\indent The tracks are projected to the Ecal and matched to clusters based on position and momentum. There exists an established parameterization of the track-cluster matching that can be used to measure the quality of the match as a a count of standard deviation, $n\sigma$. Furthermore, the Ecal has a superior timing resolution to the SVT and the time difference between two clusters can be used to study both accidentals and elimination of tracks originating from different events. \\
\indent There are a few cuts that originate specifically from studies of high z background events and physics processes which include cuts on the positron track's distance of closest approach in the $XZ$ plane to the target (DOCA), the $e+e-$ momenutm asymmetry, and the number of hits shared between tracks. All of these cuts will be discussed in detail for the 0.5~mm L1L1 dataset and their adaptations will be summarized for other datasets. Additionally, all data described here uses the tweakPass6 results. 
