 The cuts used on the L1L1 0.5~mm data set are shown in Table~\ref{tab:l1l1_cuts}.

\begin{table}[htb]
\caption{Cuts applied to the L1L1 datasets.}
\label{tab:l1l1_cuts}
\centering
\begin{tabular}{lllc}
\toprule
%\multicolumn{2}{c}{Name} \\
%\cmidrule(r){1-2}
Cut type & Cut & Cut Value &  $\%$ removed\\
\midrule
track & Fit quality & track $\chi^{2}<30$ & 60 \\
track & Max track momentum &  $P_{trk}<75\%E_{beam}$ & 11 \\
track & Isolation &   & 4  \\
vertex & beamspot constraint & bsc$\chi^{2}<10$  & 26  \\
vertex & beamspot - unconstrained & bsc$\chi^{2}$-unc$\chi^2<5$  & 9 \\
vertex & maximum $P_{sum}$ &  $<115\%E_{beam}$ & 1 \\
ECal & ECal SVT matching & $\chi^2<10$  & 7  \\
ECal & track ECal timing & $<4$ns  & 4  \\
ECal & 2 cluster time diff & $<2$ns  & 6  \\
physics & momentum asymmetry & $<0.4$  & 13 \\
physics & e+ track d0 & $<1.5$mm  & <1 \\
event & max shared hits amongst tracks & $<5$ shared hits  & 14  \\
\bottomrule
\end{tabular}
\end{table}

In Table~\ref{tab:l1l1_cuts}, the "Cut type" is a summary of what the cut is intended to have the most significant effect on. The column under "Cut" describes the cut used, and the corresponding value is annotated under the next column, "Cut Value". The "$\%$ removed" column gives a percentage of the number of events removed, from applying this cut. \\
\indent The effects of the cuts on the z vertex for all masses is shown in Figure~\ref{fig:l1l1_zvtx} in the cumulative order in which the cuts are applied. The initial track fit $\chi^{2}$ from the GBL fit of the track removes a lot of background and begins to really shape the vertex distribution. The next significant cut is the beamspot constrained $\chi^{2}$ cut. The beamspot constrained $\chi^{2}$ includes both the closest approach of the two tracks and the momentum projection of the vertex back to the beamspot position at the target. The momentum asymmetry is a cut that is primarily designed to remove WAB contributions to the data. Heavy photon generated $e^+e^-$ pairs are generally close in energy whereas the electron in WAB typically carries a much larger energy than the counterpart photon. 
\indent The last cut listed in Table~\ref{tab:l1l1_cuts} removed events where the $e^+e^-$ individual tracks share five hits with other tracks. When studying the original high $z$ background from the pass6 dataset, it was clear that nearly all of the high $z$ events were poorly reconstructed tracks due to missing hits in Layer 2 or five hits with other tracks in the event having nearly the same momentum. The cuts cleaned up all of the original high $z$ background. The effects of the cuts on the z vertex distribution can be seen in Figure~\ref{fig:l1l1_vtx}.

\begin{figure}[htb]
  \centering
      \includegraphics[width=0.75\textwidth]{pics/searching/L1L1_zvtx.png}
  \caption[Cut effects on the $z$ vertex distribution]{Cut effects on the $z$ vertex distribution for all masses in the L1L1 0.5~mm dataset.}
  \label{fig:l1l1_vtx}
\end{figure} 

The ratio of the final events selected after applying all cuts to the original events selected after initial cuts is shown in Figure~\ref{fig:l1l1_vtxR}.

\begin{figure}[htb]
  \centering
      \includegraphics[width=0.65\textwidth]{pics/searching/ratio_zvtx_cuts.png}
  \caption[Ratio of final to initial $z$ vertex events]{The ratio of the $z$ vertex distribution in the final event selection to those events in the initial event selection in the L1L1 0.5~mm dataset.}
  \label{fig:l1l1_vtxR}
\end{figure} 

The two cluster time difference can be used to study the effects of cuts on accidentals as well as the contamination of accidentals in the final sample. The evenly space 2~ns peaks as apparent in Figure~\ref{fig:l1l1_tdiff} are due to the intrinsic 499~MHz frequency at which electron beam bunches enter Hall B and interact with the HPS target. 

\begin{figure}[htb]
  \centering
      \includegraphics[width=0.8\textwidth]{pics/searching/L1L1_tdiff.png}
  \caption[Cut effects on the time difference between two clusters]{Cut effects on the two cluster time difference distribution for the L1L1 0.5~mm dataset.}
  \label{fig:l1l1_tdiff}
\end{figure} 

The effect of the cuts as shown in Figure~\ref{fig:l1l1_tdiff} on the final and initial distributions can be seen as a ratio in Figure~\ref{fig:l1l1_tdiffR}.

\begin{figure}[htb]
  \centering
      \includegraphics[width=0.65\textwidth]{pics/searching/ratio_tdiff_cuts.png}
  \caption[Ratio of final to initial events, cluster time difference]{The ratio of the two cluster time difference distribution in the final event selection (without the $\pm2$~ns cut) to those events in the initial event selection for the L1L1 0.5~mm dataset.}
  \label{fig:l1l1_tdiffR}
\end{figure}

After all cuts are applied, the accidental contamination is less than 1$\%$ in the $\pm$ 2~ns event selection (this cut is the only one not shown in Figure~\ref{fig:l1l1_tdiff}). Further studies to identify the production of high $z$ vertices using accidentals is discussed later on in this section. The cut effects on the mass distribution for the vertex search can be seen in Figure~\ref{fig:l1l1_mass}. In particular, the track-cluster matching removes the low end tail of the mass distribution which is consistent with the geometric acceptance of the experimental setup.

\begin{figure}[htb]
  \centering
      \includegraphics[width=0.75\textwidth]{pics/searching/mass_L1L1_cuts.png}
  \caption[Cut effects on the mass distribution]{Cut effects on the mass distribution for the L1L1 0.5~mm dataset.}
  \label{fig:l1l1_mass}
\end{figure} 

The initial selection for the L1L1 dataset requires that both tracks have hits in Layer 1 of the SVT and Layer 2. Previously, the requirement of Layer 2 was not used, but as the rates are highest in Layer 1 of the SVT, the extrapolation from Layer 3 to Layer 1 is critical in order to correctly measure the vertex of the track. Additionally, the inefficiency in measuring a hit in Layer 2 is approximately 2$\%$ of the time (and is different for electrons and positrons). Included in the initial event selection is also the radiative cut at 80$\%$ of the beam energy. After these initial cuts, we apply the track selection cuts. To ensure general track quality, the track $\chi^2$ is used as shown in Figure~\ref{fig:trkChi2}.

\begin{figure}[htb]
  \centering
      \includegraphics[width=0.65\textwidth]{pics/searching/trkChi2.png}
  \caption[Track $\chi^2$ cut]{The electron and positron track $\chi^2$ are shown with the cut indicated by the red line. This cut is an initial track selection quality cut and uses no vertex or timing information.}
  \label{fig:trkChi2}
\end{figure} 

After choosing tracks based on their individual fit qualities, we remove electron tracks that have greater than 75$\%$ of the beam energy. This cut is made to ensure that we are not choosing elastically-scattered (beam energy) electrons and corresponds also to the general maximum value we can expect for an electron track in trident events. A comparison between the data and Monte Carlo is shown in Figure~\ref{fig:emTrkPmax}.

\begin{figure}[htb]
  \centering
      \includegraphics[width=0.65\textwidth]{pics/searching/emTrkPmax.png}
  \caption[Maximum $e-$ track momentum]{The maximum momentum of the electron track as shown in data (prior to cutting) and Monte Carlo. Events with higher electron energies are attributable to other types of backgrounds in the data such as elastics and wide angle bremsstrahlung and not the trident background.}
  \label{fig:emTrkPmax}
\end{figure} 

The next cut applied is the isolation cut. The isolation value for the electron and the positron in the L1L1 dataset is the distance to the next closest hit away from the beamline in Layer 1 relative to the electron and positron hit used in the track. This cut compares the isolation value (parameter $\delta$) to the track projected value in y at the target position (also known as the track $z0$ parameter). If the projected isolation to the target is larger than the $z0$ parameter at the target, then we assume that the better hit was already chosen for the track. A picture of the variables used in this cut is shown in Figure~\ref{fig:isoPic}.

\begin{figure}[htb]
  \centering
      \includegraphics[width=0.8\textwidth]{pics/searching/isolationPic.png}
  \caption[Track isolation cut]{The distance between the closest hit away from the beam in Layer 1 is compared to its projection at the target the track impact parameter, $z0$ at the target. }
  \label{fig:isoPic}
\end{figure}

The cut shown is described numerically in Equation~\eqref{eq:isolationl1}.

\begin{equation}
\label{eq:isolationl1}
2\delta+z0\times\textbf{sign($PY$)}>0
\end{equation}
 
Because the distance from Layer 2 to the target is twice the distance between Layer 1 and the target, we compare the isolation times a factor of 2 to the $z0$ parameter in order to compare the projection to the target. The $z0$ parameter is opposite in sign as compared to the y-component of the track momentum because we only consider downstream vertices. 


\indent In identifying downstream vertices, we use the unconstrained vertex collection to optimize our search for detached vertices. For each vertexed $e^+e^-$ pair, we can see how the vertex is changed when different additional constraints are applied to the vertex. The unconstrained vertex collection only looks at the distance of closest approach between the two tracks. The target constrained vertex collection is optimized for a bump hunt analysis and requires that the vertex of the $e^+e^-$ pairs occurs at the target. The beamspot constrained vertex collection requires that the momentum of the vertexed pairs projects back to the beamspot location at the target and considers the distance of closest approach between the two tracks. 

The beamspot constrained vertex $\chi^2$, can give us information about how well the vertex momentum points back to the beamspot location at the target. The beamspot constraint is particularly useful in identifying events where a track has scattered significantly because the the projected momentum misses the beamspot location at the target. Real signal will always project back to the beamspot. The effect of the beamspot constraint cut on the vertex  $\chi^2$ distribution alone can be seen in Figure~\ref{fig:bsccut}.

\begin{figure}[htb]
  \centering
      \includegraphics[width=0.6\textwidth]{pics/searching/bscCut.png}
  \caption{The effect of a cut on the beamspot constrained $\chi^2$ on the vertex distribution for all masses. While this plot is shown for all masses, the effects of the cut on the tails of the distribution can still be seen. The cut removes events where tracks did not pass close to eachother in space to generate a vertex and/or the vertex does not point back to the beam position at the target.}
  \label{fig:bsccut}
\end{figure} 

The difference between the beamspot constrained and unconstrained $chi^2$ can also be used to exclusively identify how well a vertex points back to the target. The effect of this cut can be seen in Figure~\ref{fig:bmucut}.

\begin{figure}[htb]
  \centering
      \includegraphics[width=0.6\textwidth]{pics/searching/bmuchi2.png}
  \caption[Vertex cut on the difference between beamspot and unconstrained $\chi^2$]{The effect of a cut on the difference between the beamspot and unconstrained constrained $\chi^2$ on the vertex distribution for all masses. The effects of the cut on the downstream tails of the distribution tells us how well a vertexed pair of tracks points back to the beamspot position at the target.}
  \label{fig:bmucut}
\end{figure} 

The next cut is the maximum momentum of the $e^+e^-$ pair. This cut removes very few events but is still necessary to ensure that we are not including events that are not correlated. After having established the quality of the tracks and vertex using the SVT information only, the tracks are projected to their positions at the ECal and the quality of the matching of the track and ECal cluster is described as a function of the track momentum and position at the ECal. This parameter is a function of the number of $\sigma$ from distributions studied used 2015 data. The matching parameter and relevant cut value are shown in Figure~\ref{fig:matchcut}. 

\begin{figure}[htb]
  \centering
      \includegraphics[width=0.75\textwidth]{pics/searching/matchcut.png}
  \caption[Cut on the track-cluster matching]{The matching parameter for both electrons and positrons is shown. The maximum is set to 30$\sigma$, and the cut is set to 10$\sigma$ for each particle.}
  \label{fig:matchcut}
\end{figure} 

The matching cut most significantly removes the small angle/low mass background events that we saw in Figure~\ref{fig:l1l1_mass}. The timing difference between the tracks and ECal clusters removes some out of time events, but the timing resolution on the clusters is more precise than the track time and a cut on the two cluster time difference is critical for removing accidentals. The cut can be seen in Figure~\ref{fig:cltdiff}.

\begin{figure}[htb]
  \centering
      \includegraphics[width=0.6\textwidth]{pics/searching/cltdiff.png}
  \caption[Cut on the cluster pair time difference]{The two cluster timing difference is shown with a fit to the Gaussian central part of the distribution. Additional smaller peaks can be seen in intervals of 2~ns in the tails of the distribution. The timing cut can remove these events where overlapping beam buckets generate out of time vertices.}
  \label{fig:cltdiff}
\end{figure} 

Additional cuts aimed to remove the contaminating wide angle breamsstrahlung background events include a cut on the momentum asymmetry of the two tracks and the positron $D0$, or $DOCA$ (distance of closest approach to the target in the x-z plane), are used. The $e^+e^-$ pairs produced in radiative trident processes from heavy photon decays are closely symmetric in momentum. The momentum asymmetry is defined as the momentum difference of the two particles divided by the momentum sum. The electron in wide-angle bremsstrahlung typically carries a significantly higher portion of the beam energy when compared to the positron that is produced from the pair production of the lower energy photon. The effects of the momentum asymmetry cut on the vertex distribution are shown in Figure~\ref{fig:pasycut}.

\begin{figure}[htb]
  \centering
      \includegraphics[width=0.6\textwidth]{pics/searching/pasycut.png}
  \caption[Cut on the momentum asymmetry]{The effect of the momentum asymmetry cut on the tails of the vertex distribution can be seen in the red an green curves.}
  \label{fig:pasycut}
\end{figure} 

The momentum asymmetry cut can effectively reduce contamination from wide-angle bremsstrahlung in the final event sample. In wide-angle bremsstrahlung, the scattered electron and the positron from pair conversion are detected. When a positron is produced downstream of the target in this process, the projected $DOCA$ of the resultant track to the target will be positive due to the curvature of the positron track in the magnetic field from starting downstream. The cut on the $DOCA$ and its effect on the vertex distribution can be seen in Figure~\ref{fig:docacut}. 

\begin{figure}[htb]
  \centering
      \includegraphics[width=0.75\textwidth]{pics/searching/docacut.png}
  \caption[Cut on the postiron $DOCA$]{Positrons that are produced from the photon in wide-angle bremsstrahlung events will have a distance of closest approach that will curve widely at the target location, yielding a largely positive value.}
  \label{fig:docacut}
\end{figure} 

The final cut on tracks was derived after studies of the high $z$ background events showed a higher probability of one or both of the tracks sharing five hits with another track in the event. In these cases, the track with the best $\chi^2$ track fit is selected, but the momentum difference with the other track with which it shares five hits is quite small. The momentum difference of the track with the other tracks that share hits is shown versus the number of hits shared between the tracks in Figure~\ref{fig:trkshare}.

\begin{figure}[htb]
  \centering
      \includegraphics[width=0.75\textwidth]{pics/searching/TrkShareHits.png}
  \caption[Cut on shared hits between tracks]{This plot shows that many of the tracks sharing 4 and 5 hits with the initial track selected in the event have nearly the same momentum.}
  \label{fig:trkshare}
\end{figure} 

By removing tracks that only have a one hit difference with other tracks, the high $z$ background in the data sample is reduced.   
