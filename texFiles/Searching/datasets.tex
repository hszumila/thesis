During the Engineering Run, the beam energy was 1.056~GeV, and a total of 1.7~days (1166~nb$^{-1}$) of data-taking was achieved with the SVT at the nominal position where Layer 1 is at $\pm$0.5~mm from the beam. An additional 0.47~days (362.7~nb$^{-1}$) was achieved, prior to moving the SVT to the nominal position, with the SVT Layer 1 at $\pm$1.5~mm from the beam. A large portion of the data taken with the first layer of the SVT at $\pm$1.5~mm was unable to be used due to an incorrect timing latency in the SVT DAQ and is excluded from this analysis. There are a total of six datasets as shown in Table~\ref{tab:datasets}.

\begin{table}[htb]
\caption{Vertexing Datasets}
\label{tab:datasets}
\centering
\begin{tabular}{lllr}
\toprule
%\multicolumn{2}{c}{Name} \\
%\cmidrule(r){1-2}
Datasets &First hit of track & SVT position & statistics \\
\midrule
L1L1 & Both tracks layer 1 & 0.5~mm & 13,697,082\\
L1L2 & One track layer 1 & 0.5~mm & 302,103\\
L2L2 & Both tracks layer 2 & 0.5~mm & 4,876\\
L1L1 & Both tracks layer 1 & 1.5~mm & 1,635,172\\
L1L2 & One track layer 1 & 1.5~mm & 1,005,668\\
L2L2 & Both tracks layer 2 & 1.5~mm & 233,388\\
\bottomrule
\end{tabular}
\end{table}

The backgrounds, statistics, efficiencies and $zCut$s are different for each dataset, and therefore require separate analyses before combining the limits of the final results of each. The datasets are determined by the first hit layer of the $e^+e^-$ tracks that makeup the vertex. Each dataset is exclusive of the other datasets. When one hit misses Layer 1, a fiducial cut to exclude the active region of Layer 1 is used on the extrapolated track position to Layer 1 in order to ensure that the Layer 1 inefficiencies, which are still currently not completely understood, do not inflate the statistics of the dataset. 