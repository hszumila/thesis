The heavy photon, also known as the $A^{\prime}$, is a theoretically motivated massive gauge boson that is associated with a predicted U(1) hidden symmetry, favorable to Beyond Standard Model theories. According to theory~\cite{holdom_two_1986}, a heavy photon kinetically mixes with the Standard Model photon through a loop-level effect generating an effective coupling to electric charge. The relative size of this coupling to electric charge, $\epsilon$, can range from $10^{-12}$ to $10^{-2}$ depending on the loop order of the mixing interaction and describes the coupling of the heavy photon to electric charge to be at a scale significantly smaller than that in standard electrodynamic theory. While the coupling strength of the interaction can be naturally generated from the loop interactions, the mass is somewhat less constrained. Theories of the heavy photon as a way to explain cosmological phenomena make them the simplest and possibly leading interaction between the Standard Model and the Dark Sector. The Dark Sector encompasses both dark matter and dark energy particles that do not interact other than gravitationally. If the heavy photon obtains its mass through the Higgs mechanism, the mass is favored to be in the range of MeV to GeV which is compatible with dark matter theories. In such a scenario, electrons could radiate heavy photons as they do ordinary photons but at a suppressed rate. These heavy photons will have measurable lifetimes before decaying to charged particle pairs. It is natural to describe the heavy photon parameter space in terms of its coupling, $\epsilon^2$, and mass, $m_{A'}$.\\
\begin{figure}[htb]
  \centering
      \includegraphics[width=0.7\textwidth]{pics/intro/reach_nominal.png}
  \caption[Reach for the HPS experiment]{The existing 90$\%$ confidence limits from other experiments looking for heavy photons in the relevant mass-coupling region is shown. The shaded blue region includes the preliminary bump hunt results from the 2015 engineering run. The vertex reach is not shown on this plot as no reach is found using the proposed HPS run configuration. The region labeled as $a_\mu$ indicates the favored parameter space for a visibly decaying heavy photon to explain the discrepancy between the calculated and measured muon anomalous magnetic moment. The experiments along the top of the plot with large coupling look for heavy photons that decay promptly at the target. The limits shown in grey along the left side of the plot with decreasing values of coupling look for heavy photons with displaced vertices in beam dump experiments.}
      \label{Figure:projectedReach}
\end{figure}
\indent The Heavy Photon Search (HPS) experiment is searching for heavy photons in the mass range of 20 to 1000~MeV/c$^2$ with prompt or displaced vertices with respect to the target interaction. HPS  generates heavy photons from an electron beam incident on a heavy target and measures the momentum and vertex position of $e^+e^-$ pairs produced from its decay. By reconstructing the invariant mass and the vertex position of the pairs, HPS can look for a small bump on a large background using a bump hunt for prompt decays. Uniquely, HPS is also able to look for heavy photons with smaller couplings (and longer lifetimes) characterized by displaced vertices by searching for a small signal on low background downstream of the target. The HPS reach attained from the 2015 engineering run from the bump hunt is shown in Figure~\ref{Figure:projReach} along with the existing limits from other experiments.\\
\indent The HPS experiment took place in Hall B at the Jefferson Laboratory National Accelerator Facility. The Continuous Electron Beam Accelerator Facility (CEBAF) at Jefferson Lab produces an electron beam that collides with the HPS target material in Hall B. The HPS detector measures the particles from this interaction and searches for the heavy photon signal. The HPS detector consists of a Silicon Vertex Tracker (SVT) and an Electromagnetic Calorimeter (ECal). The SVT measures particle trajectories and reconstructs the vertex position of the particle pair. The ECal triggers event readout in addition to measuring particle energy and pair coincidence timing. \\
\indent The ECal was commissioned during a short commissioning run in December 2014. The full experiment ran in the spring of 2015 commissioning the full beamline and both detectors. This run took 2.3~days of good data at approximately 50~nA with a beam energy of 1.056~GeV. HPS obtained a total of 1529~nb$^{-1}$ of good data. During the commissioning of the SVT, some data was taken with the SVT slightly open from its nominal position before moving the SVT in to its designed position at $\pm0.5$~mm from the beam. A second run in the spring of 2016 used a 200~nA electron beam at 2.3~GeV collecting a total of 5.7~days of data.  Future running at higher electron beam energy is planned for 2018 and beyond.\\
\indent In this dissertation, I describe the search for heavy photons with a displaced vertex using data from the 2015 engineering run. I will describe the experiment as a whole focusing on the areas in which I was most involved. I performed a blinded analysis using 10$\%$ of the data. In order to better understand and analyze the backgrounds in the vertex search, I conducted a further study of the backgrounds using the statistics of the fully unblinded dataset. I will discuss the backgrounds and reach from the Engineering Run.\\
\indent In addition to the full vertex analysis, I contributed significantly to the assembly, characterization and commissioning of the ECal for all experimental running. I wrote the clustering algorithm based on that used by the CLAS experiment Inner Calorimeter (IC) and improved simulations of the ECal detector response. I also calibrated the ECal in both energy and time for both experimental runs. 