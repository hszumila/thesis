\subsection{L1L2 with SVT at 0.5 mm}

The L1L2 data set with the SVT at the nominal 0.5~mm position consists of tracks where the one track missed the active region of Layer 1. This data set combines the case for which the electron passes through Layer 1 and the positron passes through Layer 1 in order to obtaining the $zCut$ because it was noted that the tails of the distributions are the same (despite the known backgrounds being different and could merit improved cuts that would divide the data set in the future). \\
\indent The cuts applied to the L1L2 data set are shown in Table~\ref{l1l2_cuts}. 

\begin{table}[H]
\caption{Cuts applied to the L1L2 datasets.}
\label{l1l2_cuts}
\centering
\begin{tabular}{lllllll}
\toprule
%\multicolumn{2}{c}{Name} \\
%\cmidrule(r){1-2}
Cut type & Cut & Cut Value &  $\%$killed &  $\%$killed core & $\%$killed tails\\
\midrule
track & Fit quality & track $\chi^{2}<30$ & 38 & 15 & 47 \\
track & Max track momentum &  $P_{trk}<75\%E_{beam}$ & 12 & 8 & 14 \\
track & Isolation &   & 11 & 4 & 15 \\
vertex & beamspot constraint & bsc$\chi^{2}<10$  & 46 & 24 & 60 \\
vertex & beamspot - unconstrained & bsc$\chi^{2}$-unc$\chi^2<5$  & 20 & 16 & 24 \\
vertex & maximum $P_{sum}$ &  $<115\%E_{beam}$ & 1 & 1 & 1 \\
ecal & Ecal SVT matching & $\chi^2<10$  & 7 & 7 & 8 \\
ecal & track Ecal timing & $<4$ns  & 5 & 5 & 5 \\
ecal & 2 cluster time diff & $<2$ns  & 8 & 6 & 10 \\
physics & momentum asymmetry & $<0.4$  & 14 & 15 & 13 \\
physics & e+ track d0 & $<1.5$mm  & 7 & 3 & 11 \\
event & max shared hits amongst tracks & $<5$ shared hits  & 8 & 7 & 8 \\
track & cuts on kink tails & $\phi$ and $\lambda$ kink tails & 19 & 9 & 36 \\
\bottomrule
\end{tabular}
\end{table}

The initial selection requires that a track that missed Layer 1 has a projection to the z location at layer 1 that is less than 1.5~mm from the beam. This ensures that the sample is not overly contaminated by events that passed through the active region but failed to identify a hit. As a result, the core of the distribution sits on the downstream side of the z-axis and reflects the geometric constraints we have imposed. The first cut that is different from the L1L1 data set is the isolation cut. In this data set, we apply the same isolation cut to the track that passed through Layer 1, but we apply a slightly different isolation cut for the track that did not pass through layer 1. The isolation in layer 2 is measured and projected to the target position to be compared with the impact parameter of the track in $y$ at the target. Additional cuts are applied to the tails of the kink distributions for the tracks. The summary of these cuts is made in the Table~\ref{kink_cuts}.

\begin{table}[H]
\caption{Cuts applied to the kinks in layers 1-3.}
\label{kink_cuts}
\centering
\begin{tabular}{lll}
\toprule
%\multicolumn{2}{c}{Name} \\
%\cmidrule(r){1-2}
Cut & Value \\
\midrule
Layer 1: $\phi$ kink, $\lambda$ kink & <0.0001,<0.002\\
Layer 2: $\phi$ kink, $\lambda$ kink & <0.002,<0.004\\
Layer 3: $\phi$ kink, $\lambda$ kink & <0.002,<0.004\\
\bottomrule
\end{tabular}
\end{table}

The uncut kink distributions for the electron with the cut indicated by the red dashed line is shown in Figures~\ref{fig:kink1}, \ref{fig:kink2}, and \ref{fig:kink3}.

\begin{figure}[H]
  \centering
      \includegraphics[width=0.8\textwidth]{pics/appendix/kink1.png}
  \caption{The kink distributions for tracks passing through Layer 1. The cut is shown at the red dashed line.}
  \label{fig:kink1}
\end{figure} 
\begin{figure}[H]
  \centering
      \includegraphics[width=0.8\textwidth]{pics/appendix/kink2.png}
  \caption{The kink distributions for tracks passing through Layer 2. The cut is shown at the red dashed line.}
  \label{fig:kink2}
\end{figure} 
\begin{figure}[H]
  \centering
      \includegraphics[width=0.8\textwidth]{pics/appendix/kink3.png}
  \caption{The kink distributions for tracks passing through Layer 3. The cut is shown at the red dashed line.}
  \label{fig:kink3}
\end{figure} 

The positron kink distributions look similar to the electron kink distributions and are not shown here. These cuts remove events from the tails. The effects of all the cuts on the reconstructed vertex position distribution are shown in Figure~\ref{fig:zvtxCuts_l1l2}.

\begin{figure}[H]
  \centering
      \includegraphics[width=0.8\textwidth]{pics/appendix/zvtxCuts_L1L2.png}
  \caption{The effects of the cuts on the L1L2 dataset on the unconstrained z vertex.}
  \label{fig:zvtxCuts_l1l2}
\end{figure} 

The effects of the cuts on the reconstructed mass distribution are shown in Figure~\ref{fig:massCuts_l1l2}.

\begin{figure}[H]
  \centering
      \includegraphics[width=0.8\textwidth]{pics/appendix/massCuts_L1L2.png}
  \caption{The effects of the cuts on the L1L2 dataset on the mass distribution.}
  \label{fig:massCuts_l1l2}
\end{figure} 

This dataset has the tendency to contain more WAB contamination than the L1L1 dataset. In particular, we know from Monte Carlo that positrons are unlikely to have a hit in Layer 1 when the photon in WAB pair produces after the target. Additionally, this sample contains a 5:1 ratio of having in electron versus a positron in the first layer. 

\subsection{L2L2 with SVT at 0.5 mm}

The L2L2 dataset consists of vertices produced when tracks do not pass through Layer 1 and their projections back to Layer 1 are within 1.5~mm of the beam (outside the active silicon region). This dataset requires the most work to remove the background events, and preliminary studies with the small number of statistics have been unsuccessful. \\
\indent The general cuts applied to the L2L2 dataset, after first requiring that track projections do not extend to the active region of Layer 1, are listed in Table~\ref{l2l2_cuts}.

\begin{table}[H]
\caption{Cuts applied to the L2L2 datasets.}
\label{l2l2_cuts}
\centering
\begin{tabular}{lllllll}
\toprule
%\multicolumn{2}{c}{Name} \\
%\cmidrule(r){1-2}
Cut type & Cut & Cut Value &  $\%$killed &  $\%$killed core & $\%$killed tails\\
\midrule
track & Fit quality & track $\chi^{2}<30$ & 44 & 66 & 44 \\
track & Max track momentum &  $P_{trk}<75\%E_{beam}$ & 15 & 14 & 15 \\
track & Isolation &   & 22 & 34 & 22 \\
vertex & beamspot constraint & bsc$\chi^{2}<10$  & 47 & 36 & 47 \\
vertex & beamspot - unconstrained & bsc$\chi^{2}$-unc$\chi^2<5$  & 18 & 0 & 19 \\
vertex & maximum $P_{sum}$ &  $<115\%E_{beam}$ & 1 & 6 & 1 \\
ecal & Ecal SVT matching & $\chi^2<10$  & 30 & 73 & 29 \\
ecal & track Ecal timing & $<4$ns  & 7 & 0 & 8 \\
ecal & 2 cluster time diff & $<2$ns  & 8 & 0 & 8 \\
physics & momentum asymmetry & $<0.4$  & 4 & 0 & 4 \\
physics & e+ track d0 & $<1.5$mm  & 21 & 33 & 21 \\
event & max shared hits amongst tracks & $<4$ shared hits  & 21 & 50 & 21 \\
\bottomrule
\end{tabular}
\end{table}

The geometric acceptance of the cuts in the L2L2 data set leave a core fraction of background events well beyond the target at approximately 30~mm downstream. The only modifications to previously applied cuts are that both tracks use a modified isolation cut by looking at the isolation at Layer 2 and the tracks do not share 4 hits with any other track in the event.  The kink cuts appeared to not remove events from this data set.

\subsection{L1L1 with SVT at 1.5 mm}
The L1L1 dataset in the 1.5~mm data includes vertices reconstructed from pairs of tracks that have hits in Layer 1 of the SVT. Due to the SVT opening being larger, the acceptance favors larger heavy photon masses. The SVT has also lower rates in Layer 1 when compared to the 0.5~mm dataset.\\
\indent The cuts applied to the L1L1 dataset are shown in Table~\ref{l1l1_cuts_1p5}.

\begin{table}[H]
\caption{Cuts applied to the L1L1 datasets with the SVT at 1.5mm.}
\label{l1l1_cuts_1p5}
\centering
\begin{tabular}{llllll}
\toprule
%\multicolumn{2}{c}{Name} \\
%\cmidrule(r){1-2}
Cut type & Cut & Cut Value &  $\%$killed &  $\%$killed core & $\%$killed tails\\
\midrule
track & Fit quality & track $\chi^{2}<30$ & 37 & 22 & 87 \\
track & Max track momentum &  $P_{trk}<75\%E_{beam}$ & 6 & 6 & 19 \\
track & Isolation &   & 2 & 1 & 15 \\
vertex & beamspot constraint & bsc$\chi^{2}<10$  & 23 & 21 & 81 \\
vertex & beamspot - unconstrained & bsc$\chi^{2}$-unc$\chi^2<5$  & 12 & 12 & 27 \\
vertex & maximum $P_{sum}$ &  $<115\%E_{beam}$ & 0 & 0 & 2 \\
ecal & Ecal SVT matching & $\chi^2<10$  & 3 & 3 & 58 \\
ecal & track Ecal timing & $<4$ns  & 5 & 5 & 7 \\
ecal & 2 cluster time diff & $<2$ns  & 4 & 4 & 13 \\
physics & momentum asymmetry & $<0.4$  & 12 & 12 & 48 \\
physics & e+ track d0 & $<1.5$mm  & 0 & 0 & 4 \\
event & max shared hits amongst tracks & $<5$ shared hits  & 12 & 12 & 20 \\
\bottomrule
\end{tabular}
\end{table}

The cuts are the same as those applied to the 0.5~mm dataset with similar effect.

\subsection{L1L2 with SVT at 1.5 mm}
The following section describes the data set where one track misses Layer 1 of the SVT and its track projection back to Layer 1 is within 2.5~mm of the beam such that the track does not extrapolate to the active region of the silicon.\\
\indent The cuts applied to the L1L2 dataset with the first layer of the SVT at 1.5~mm is shown in Table~\ref{l1l2_cuts_1p5}.

\begin{table}[H]
\caption{Cuts applied to the L1L2 datasets with the SVT at 1.5~mm.}
\label{l1l2_cuts_1p5}
\centering
\begin{tabular}{lllllll}
\toprule
%\multicolumn{2}{c}{Name} \\
%\cmidrule(r){1-2}
Cut type & Cut & Cut Value &  $\%$killed &  $\%$killed core & $\%$killed tails\\
\midrule
track & Fit quality & track $\chi^{2}<30$ & 23 & 11 & 47 \\
track & Max track momentum &  $P_{trk}<75\%E_{beam}$ & 8 & 7 & 12 \\
track & Isolation &   & 4 & 2 & 10 \\
vertex & beamspot constraint & bsc$\chi^{2}<10$  & 29 & 20 & 62 \\
vertex & beamspot - unconstrained & bsc$\chi^{2}$-unc$\chi^2<5$  & 12 & 11 & 22 \\
vertex & maximum $P_{sum}$ &  $<115\%E_{beam}$ & 0 & 0 & 0 \\
ecal & Ecal SVT matching & $\chi^2<10$  & 5 & 5 & 7 \\
ecal & track Ecal timing & $<4$ns  & 5 & 5 & 5 \\
ecal & 2 cluster time diff & $<2$ns  & 6 & 5 & 9 \\
physics & momentum asymmetry & $<0.4$  & 14 & 13 & 16 \\
physics & e+ track d0 & $<1.5$mm  & 6 & 5 & 16 \\
event & max shared hits amongst tracks & $<5$ shared hits  & 6 & 6 & 6 \\
track & cuts on kink tails & $\phi$ and $\lambda$ kink tails & 22 & 8 & 74 \\
\bottomrule
\end{tabular}
\end{table}

The cuts applied to the L1L2 dataset may require a similar optimization to eliminate backgrounds as that required of the dataset for the 0.5~mm. Namely, that, it may be necessary to separate the dataset for events where the positron versus the electron is the first to leave a hit in Layer 1. For the moment, the same cuts are used as the 1.5~mm dataset has generally lower backgrounds than that seen in the 0.5~mm dataset.

\subsection{L2L2 with SVT at 1.5 mm}
The following section discusses the events having no hit in Layer 1 of the 1.5~mm dataset. An additional requirement was made that the tracks must not project back to the active region of the Layer 1 silicon in order to avoid contamination by events with the Layer 1 inefficiency. \\
\indent The cuts applied to the L2L2 dataset are shown in Table~\ref{l2l2_cuts_1p5}.

\begin{table}[H]
\caption{Cuts applied to the L2L2 datasets with the SVT at 1.5~mm.}
\label{l2l2_cuts_1p5}
\centering
\begin{tabular}{lllllll}
\toprule
%\multicolumn{2}{c}{Name} \\
%\cmidrule(r){1-2}
Cut type & Cut & Cut Value &  $\%$killed &  $\%$killed core & $\%$killed tails\\
\midrule
track & Fit quality & track $\chi^{2}<30$ & 29 & 11 & 39 \\
track & Max track momentum &  $P_{trk}<75\%E_{beam}$ & 10 & 8 & 12 \\
track & Isolation &   & 5 & 2 & 8 \\
vertex & beamspot constraint & bsc$\chi^{2}<10$  & 26 & 16 & 35 \\
vertex & beamspot - unconstrained & bsc$\chi^{2}$-unc$\chi^2<5$  & 10 & 8 & 14 \\
vertex & maximum $P_{sum}$ &  $<115\%E_{beam}$ & 1 & 1 & 1 \\
ecal & Ecal SVT matching & $\chi^2<10$  & 11 & 8 & 14 \\
ecal & track Ecal timing & $<4$ns  & 6 & 6 & 6 \\
ecal & 2 cluster time diff & $<2$ns  & 7 & 6 & 7 \\
physics & momentum asymmetry & $<0.4$  & 3 & 2 & 4 \\
physics & e+ track d0 & $<1.5$mm  & 9 & 7 & 12 \\
event & max shared hits amongst tracks & $<4$ shared hits  & 20 & 20 & 20 \\
\bottomrule
\end{tabular}
\end{table}

The cuts are the same as those applied to the data in the 0.5~mm L2L2 dataset. This dataset has significantly more statistics that needs to be studied and may complement the analysis on the 0.5~mm high z background data. 