The primary physics trigger looks for two cluster events and recorded a high yield of WAB events composed of a final state electron and photon. The spectrum of cluster energies in the Engineering Run dataset shows an excess of WAB events occurring where the energy sum of the two particles is approximately equal to the beam energy, uncorrected for shower-loss effects in the ECal.  \\
\indent Initial studies showed that the energy sum of two particles in WAB events having mid-range energies was lower than the reconstructed elastic energy, indicating that the shower loss corrections in the mid-range beam energy required further investigation. WAB events were used to refine the shower loss corrections for mid-range energy particles. WAB events are identified by having one track-matched cluster and no track matching a photon cluster. The reconstructed energy sum of of the two particles must equal the beam energy~\cite{szumila-vance_hps_2016}

\begin{equation}
	\label{eq:wabBeam}
	E_i = \dfrac{E_{e-}}{f_{e-}(E_{e-})}+\dfrac{E_{\gamma}}{f_{\gamma}(E_{\gamma})}
\end{equation}

where $f$ refers to the shower loss correction described by Equation~\eqref{eq:eclsf}. The underlying assumption is that the relationship between the electron and photon shower loss corrections found in Monte Carlo are preserved according to Equation~\eqref{eq:wabRatio}.

\begin{equation}
	\label{eq:wabRatio}
	 \dfrac{f_{e-, data}(E_{e-})}{f_{\gamma, data}(E_{\gamma})}= \dfrac{f_{e-, MC}(E_{e-})}{f_{\gamma, MC}(E_{\gamma})}
\end{equation}

In maintaining the relationships shown in Equations~\eqref{eq:wabBeam} and \eqref{eq:wabRatio}, a chi-squared minimization yields the optimal adjustments to the shower loss correction functions for mid-range energy particles as shown in Equation~\eqref{eq:wabBeamChi}.

\begin{equation}
	\label{eq:wabBeamChi}
	\chi^2 =\sum_{i} \dfrac{(E_{beam}-E_{sum})^2}{\sigma_{e-}^2(E_{e-})+\sigma_{\gamma}^2(E_{\gamma})}	
\end{equation}

For each event, the energy sum of the two corrected clusters, $E_{sum}$, is calculated as described by Equation~\eqref{eq:wabBeamChi}.The end result is a small correction to the shower loss correction functions that ranges across the cluster energies and never exceeds a difference of 2$\%$~\cite{szumila-vance_hps_2016}. After incorporating these updated corrections to the shower loss correction functions, the energy resolution can be extracted for all energies and positions in the ECal.  
