The large area APDs enabled the ECal to have the sensitivity to detect signals from cosmic muons traversing the ECal crystals perpendicularly. This signal was used for the initial calibration of the modules. The experimental setup was modeled using Monte Carlo simulations so that the energy deposited in the crystals from cosmic ray muons and rates could be studied. By measuring the average path length of cosmic ray muons in each crystal, the energy deposited in each crystal of the ECal was calculated using the known energy deposition from 2~GeV muons \cite{Agashe:2014kda}. 

The experimental setup for the cosmic calibrations used two scintillators placed below the ECal to trigger readout of all of the crystals. A schematic for the setup of the cosmic calibration is shown in Fig.~\ref{Figure:cosmicScheme}.

%include some discussion of the improvement achieved after removing the splitters

\begin{figure}[htb]
  \centering
      \includegraphics[width=0.5\textwidth]{pics/performance/cosmicschematic.png}
  \caption[Setup for ECal cosmic ray calibration]{Experimental setup for the cosmic ray calibration. As a cosmic ray passed vertically through both scintillators, event readout is triggered.}
  \label{Figure:cosmicScheme}
\end{figure}

Each scintillator measures 75~cm long, 22~cm wide and 5~cm thick, covering a slightly larger perpendicular area than the ECal crystals. The two scintillators are less than half a meter apart with the closest scintillator less than half a meter beneath the ECal. The energy deposited in each crystal in a layer is sensitive to the path length of the track as it passed through the crystal. From simulation, the average energy deposited in a crystal by a cosmic ray passing vertically though the ECal is shown in Fig.~\ref{Figure:cosmicEdep}.

\begin{figure}[htb]
  \centering
      \includegraphics[width=0.8\textwidth]{pics/performance/cosmicEdep.png}
  \caption[Simulation of energy deposited per ECal module from cosmic rays]{The simulated cosmic ray muon energy deposition per crystal of the ECal. The mean energy was 18.3~MeV.}
  \label{Figure:cosmicEdep}
\end{figure}

In Fig.~\ref{Figure:cosmicEdep}, only tracks passing through one crystal in each row were included. Additionally, the cosmic ray muon track had to pass through the crystals immediately above and below the struck crystal. For crystals near edges, the geometrical requirement was adjusted to include the two crystals immediately above (or below for cases where the edge is above the crystal) the crystal being readout. The average energy deposited per crystal is approximately 18.3~MeV~\cite{Agashe:2014kda}
In data, the raw FADC waveform for each crystal is readout, and the event is kept for further study after applying strict coincidence cuts between the two scintillators. The trigger rate for data is about 7~Hz. 30$\%$ of events passed the coincidence cut between the scintillators. For a track passing vertically through all ten layers, we can see the signal in each crystal as shown in Fig.~\ref{Figure:cosmicSig}.

\begin{figure}[htb]
  \centering
      \includegraphics[width=0.7\textwidth]{pics/performance/cosmicSignal.png}
  \caption[Real cosmic ray signal in raw FADC waveform passing vertically through ECal]{Cosmic ray signal passing vertically through all ten layers of crystals in the ECal. Each crystal's signal is separated vertically in this plot by its pedestal. The arrow indicates the approximate time that the cosmic signal passed through the detector.}
  \label{Figure:cosmicSig}
\end{figure}

As seen in Fig.~\ref{Figure:cosmicSig}, each FADC channel has a different pedestal value. The pedestal for each event was calculated as an average of the first twenty bins of the time window. By searching for a threshold crossing in the time window where cosmic events occurred, the signal was then fully integrated and the pedestal was subtracted. The raw waveform thresholds were 2.5~mV in 2015 and increased to 3.5~mV in 2016 to accommodate the larger signals after the removal of the splitters. As shown in Figure~\ref{readoutChain}, the $1/3$ of the signal from the ECal was split to go to the TDC modules. The splitters and TDCs were removed for the 2016 Physics Run so that the full signal could be measured by the FADC. Geometric cuts are then applied to the data in offline analysis. Crystals having peaks over a certain threshold must have at least an adjacent crystal located above and below with threshold crossing, but the crystals to the left and right must not cross threshold. These cuts ensure that that the track passed as vertically as possible through the ECal (reducing the variations in path length across each crystal). The integrated signals over many events in each crystal were fit. An individual crystal fit is shown in Fig.~\ref{Figure:cosmicFit}.

\begin{figure}[htb]
  \centering
      \includegraphics[width=0.8\textwidth]{pics/performance/cosmicFitExample2015.pdf}
  \caption[Integrated cosmic signal in ECal fitted for calibration]{Each cosmic signal was integrated and then fit using a Landau-Gaussian convolution function. The peak was calculated numerically from this fit.}
  \label{Figure:cosmicFit}
\end{figure}

The fit shown in Fig.~\ref{Figure:cosmicFit} utilized a Landau-Gaussian convolution as the Landau part corresponds to the crystal's response to a particle's energy deposition, and the Gaussian part accounts for the statistical nature of the electronics shaping and readout. The peak of the fit is calculated numerically, and the initial conversion from units of FADC to energy (MeV) is obtained (called the Gain factor). 

1~V corresponds to 4096~FADC channels. The 4096~FADC counts can be set to 1~V or 2~V, but for 2015 and 2016 running, the setting was 1~V. The gain factor is calculated using the measured peak position in units of FADC and the known energy deposited from the simulation in units of MeV as shown in Eqn.~\eqref{eq:gain}.

\begin{equation}
	\label{eq:gain}
	\textrm{Gain} = \dfrac{\textrm{[MeV]}}{\textrm{[FADC]}} 
\end{equation}

After approximately 60~hours of cosmic data, the full ECal could be calibrated, and the resultant gains for all channels in the Engineering Run are shown in Figures.~\ref{Figure:cosmicG} and~\ref{Figure:cosmicGhisto}.

\begin{figure}[htb]
  \centering
      \includegraphics[width=0.7\textwidth]{pics/performance/cosmicGains2015.png}
  \caption[Resulting 2015 gain calibration in the ECal using cosmic ray muons shown by ECal module position]{Resulting gain calibration for each crystal using cosmics for the Engineering Run where the $z-$axis color is shown in units of [MeV/FADC].}
  \label{Figure:cosmicG}
\end{figure}


\begin{figure}[htb]
  \centering
      \includegraphics[width=0.7\textwidth]{pics/performance/cosmicGainsMay15.pdf}
  \caption[Distribution of the resulting 2015 gains in the ECal using cosmic ray muons]{Resulting gain calibration for all crystals using cosmics for the Engineering Run.}
  \label{Figure:cosmicGhisto}
\end{figure}

With the splitters installed in the ECal readout chain (shown in Figure~\ref{Figure:readoutChain}, the average gain value was around 0.2~MeV/FADC for the Engineering Run. After the removal of the splitters in January 2016, prior to the Physics Run, the average gains were found to be around 0.13~MeV/FADC.
