The HPS experiment sends an electron beam onto a thin tungsten target and looks for radiated heavy photons in the reconstructed $e+e-$ mass spectrum. HPS looks for heavy photons in the range of 20 to 1000~MeV/c$^2$ and covers this territory with two searches on the same dataset that probe different heavy photon coupling regimes. A bump hunt searches for the heavy photon signal as a resonance on a large background. The bump hunt looks for heavy photons with large couplings and decay at the target. The vertex search looks for heavy photons that have detached vertices, having a measurable lifetime and decaying downstream of the target. 

\subsection{Signal}

The heavy photon is generated from the electron beam interaction with a heavy target as shown in Figure~\ref{fig:apTree} where $Z$ is the atomic number corresponding to the target material.  

\begin{figure}[H]
    \begin{center}
	\begin{fmffile}{apTree}
	\begin{fmfgraph*}(150,150)
	\fmfstraight
		\fmfleft{i1,i2,i3,i4}
		\fmfright{o1,o2,o3,o4}
		\fmflabel{$Z$}{i1}
		\fmflabel{$e-$}{i2}
		\fmflabel{$e-$}{o2}
		\fmflabel{$e-$}{o3}
		\fmflabel{$e+$}{o4}	
		\fmf{heavy}{i1,v1,o1}
		\fmffreeze
		\fmf{fermion}{i2,v2,v3,o2}				
		\fmffreeze	
		\fmf{fermion}{o4,v4,o3}
		\fmf{photon,tension=0,label=$\gamma$}{v1,v2}
		\fmf{photon,tension=2,label=$A^{\prime}$}{v3,v4}
	\end{fmfgraph*}
	\end{fmffile}
  	\end{center}
    	\caption[Heavy photon production in a fixed-target experiment]{The heavy photon is produced in a process analagous to bremsstrahlung on a heavy target of atomic number $Z$.}
   	 \label{fig:apTree}	
\end{figure}

Shown for the HPS experiment, the heavy photon decays to $e+e-$ pairs with a measurable mass and possible displaced vertex downstream from the target. The differential cross section for heavy photon production is described in terms of the electron beam incident energy $E_0$, heavy photon mass $m_{A^{\prime}}$, and fraction of incident beam energy carried by the the heavy photon $x\equiv E_{A^{\prime}}/E_0$ in Equation~\eqref{eq:apDiffCS}~\cite{bjorken_new_2009}. 

\begin{equation}
	\label{eq:apDiffCS}
	\dfrac{d\sigma}{dxd\cos\theta_{A^{\prime}}} \approx \dfrac{8Z^2\alpha^3\epsilon^2E_0^2x}{U(x,\theta_{A^{\prime}})^2}\log\Big( (1-x+\dfrac{x^2}{2})-\dfrac{x(1-x)m_{A^{\prime}}^2E_0^2x\theta_{A^{\prime}}^2}{U(x,\theta_{A^{\prime}})^2}\Big)
\end{equation}

In Equation~\eqref{eq:apDiffCS}, $Z$ is the atomic number of the target material, $\alpha$ is the usual fine structure constant, $\theta_{A^{\prime}}$ is the lab frame angular difference between the incoming electron and outgoing heavy photon. The virtuality of the intermediate electron is described by Equation~\eqref{eq:virtuality}.

\begin{equation}
	\label{eq:virtuality}
	U(x,\theta_{A^{\prime}}) = E_0^2x\theta_{A^{\prime}}^2+m_{A^{\prime}}^2\dfrac{1-x}{x}+m_e^2x 
\end{equation}

where $m_e$ is the mass of the electron. The cross section is further simplified for $m_e\ll m_{A^{\prime}}\ll E_0$ and $x\theta_{A^{\prime}}^2\ll 1$. Integrating Equation~\eqref{eq:apDiffCS} over the angle, the cross section is shown in Equation~\eqref{eq:csFinal}.

\begin{equation}
	\label{eq:csFinal}
	\dfrac{d\sigma}{dx} \approx \dfrac{8Z^2\alpha^3\epsilon^2x}{m_{A^{\sigma}}^2}\Big(1+\dfrac{x^2}{3(1-x)}\Big)
\end{equation}

The total heavy photon production rate is controlled by $\alpha^3\epsilon^2/m_{A^{\prime}}^2$ and is suppressed to photon bremsstrahlung by $\epsilon^2m_e^2/m_{A^{\prime}}^2$.The singularity is regulated by the mass of the electron and cutoff for values where $1-x$ exceeds $m_e^2/m_{A^{\prime}}^2$ or $m_{A^{\prime}}^2/E_0^2$. The heavy photon signal carries nearly the entire beam energy such that the median value of $1-x\sim\textsf{max}\Big(\dfrac{m_e}{m_{A^{\prime}}}, \dfrac{m_{A^{\prime}}}{E_0}\Big)$. The heavy photon is emitted predominately at small angles with a cutoff at $\dfrac{m_{A^{\prime}}^{3/2}}{E_0^{3/2}}$ such that the angular emission falls off as $1/\theta_{A^{\prime}}^4$.\\
 \indent The heavy photon signal is characterized by its mass (as reconstructed from the decay to $e+e-$) and decay length. Depending on the coupling strength $\epsilon$, the vertex may be reconstructed from a prompt decay at the target or a measurable decay downstream.  

\subsection{Backgrounds}

The primary backgrounds in this experiment include trident events and wide angle bremsstrahlung (WAB). The tridents are broadly categorized into radiative and Bethe-Heitler diagrams are characterized by a three particle final state $e-e-e+$~\cite{bjorken_new_2009}. The trident events were the primary source of background considered prior to the running of the experiment. It was found after experimental running that WAB events contributed to the background with an $e-\gamma$ final state where the photon pair produced to $e+e-$. In most cases, the event was triggered by the initial electron and pair-produced positron. 

\begin{figure}[H]
    \begin{center}
	\begin{fmffile}{radTree}
	\begin{fmfgraph*}(150,150)
	\fmfstraight
		\fmfleft{i1,i2,i3,i4}
		\fmfright{o1,o2,o3,o4}
		\fmflabel{$Z$}{i1}
		\fmflabel{$e-$}{i2}
		\fmflabel{$e-$}{o2}
		\fmflabel{$e-$}{o3}
		\fmflabel{$e+$}{o4}	
		\fmf{heavy}{i1,v1,o1}
		\fmffreeze
		\fmf{fermion}{i2,v2,v3,o2}				
		\fmffreeze	
		\fmf{fermion}{o4,v4,o3}
		\fmf{photon,tension=2,label=$\gamma$}{v1,v2}
		\fmf{photon,tension=2,label=$\gamma$}{v3,v4}
	\end{fmfgraph*}
	\end{fmffile}
  	\end{center}
    	\caption[Radiative background]{The radiatives have the same kinematics as the heavy photon and comprise the primary background in the bump hunt analysis where all decays are prompt. The photon is radiated from the electron incident on a heavy target of atomic number $Z$ and produces and $e+e-$ pair.}
   	 \label{fig:radTree}	
\end{figure}

The radiative background is irreducible and comprises the smooth background upon which the bump hunt search for the heavy photon signal is conducted. The Bethe-Heitler tridents also contribute significantly to the background at high energy sum although they are peaked at low energy sum. The Bethe-Heitler contribution is shown in Figure~\ref{fig:bhTree}.

\begin{figure}[H]
    \begin{center}
	\begin{fmffile}{bhTree}
	\begin{fmfgraph*}(150,150)
	\fmfstraight
		\fmfleft{i1,i2,i3,i4}
		\fmfright{o1,o2,o3,o4}
		\fmflabel{$Z$}{i1}
		\fmflabel{$e-$}{i4}
		\fmflabel{$e-$}{o2}
		\fmflabel{$e+$}{o3}
		\fmflabel{$e-$}{o4}	
		\fmf{heavy}{i1,v1,o1}
		\fmffreeze
		\fmf{fermion}{i4,v4,o4}				
		\fmffreeze	
		\fmf{photon,tension=1,label=$\gamma$}{v1,v2}
		\fmf{photon,tension=1,label=$\gamma$}{v3,v4}
		\fmf{fermion}{o3,v3,v2,o2}
	\end{fmfgraph*}
	\end{fmffile}
  	\end{center}
    	\caption[Bethe-Heitler background]{The Bethe-Heitler process has a higher production higher than the radiatives but can be reduced by requiring that the energy sum of the $e+e-$ pair be close to the beam energy. Additional contributions are considered when the recoil electron and the Bethe-Heitler positron are detected.The electron interaction with a target of atomic number $Z$ is shown above.}
   	 \label{fig:bhTree}	
\end{figure}

The trident backgrounds also produce interferences between the radiative and Bethe-Heitler diagrams although these generally only contribute at an $e+e-$ pair energy sum much less than the beam energy. 