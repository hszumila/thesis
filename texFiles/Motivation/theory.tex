The postulation for the heavy photon rests of an exploitation of allowable symmetries from the Standard Model. An additional U(1) symmetry in nature could interact with the SM through the mechanism of kinetic mixing~\cite{Holdom0}. In this model, the charge of SM particles would be shifted by some amount $\epsilon$ that is related to the strength of the new gauge boson (heavy photon or A$^{\prime}$ coupling to the SM charge. Kinetic mixing generates a coupling, $\epsilon$ through loop interactions as shown in Figure~\ref{fig:loop}. 

\begin{figure}[H]
    \begin{center}
        \begin{fmffile}{loop}
            \begin{fmfgraph*}(150,150)
                \fmfstraight 
                \fmfleft{i1}
                \fmfright{o1}
                \fmflabel{$\gamma$}{i1}
                \fmflabel{$A'$}{o1}
                \fmf{photon,tension=1}{i1,v1}
                \fmf{photon,tension=1}{v2,o1}
                \fmf{fermion,left,label=$\chi$}{v2,v1}
                \fmf{fermion,left,label=$\chi^{\prime}$}{v1,v2}
            \end{fmfgraph*}
        \end{fmffile}
    \end{center}
    \caption[Kinetic mixing of the SM photon with a heavy photon]{Kinetic mixing of the SM photon with a heavy photon is shown at the one-loop level. $chi$ can be any massive particle that is charged under both the A$^{\prime}$ and SM U(1) interactions.}
    \label{fig:loop}
\end{figure}

In the simplest scenario, there is one particle $\chi$ that is charged under both the U(1) and new U(1)$^{\prime}$. This single loop level interaction can generate the $\epsilon$ coupling to be in the range of $10^{-2}$ to $10^{-4}$. In Grand Unified Theory (GUT), symmetries forbid one-loop interactions and favor two-loop interactions generating an $\epsilon$ in the range of $10^{-3}$ to $10^{-5}$.~\cite{DarkSectors2016} If both U(1)s are in unified groups, higher loop interactions generating even smaller couplings are possible. The gauge part of the SM Lagrangian is modified to include this interaction as shown in Equation~\eqref{eq:lagrangian} 
\begin{equation}
	\label{eq:lagrangian}
\mathcal{L}_{gauge} = -\dfrac{1}{4}F^{\mu\nu}F_{\mu\nu}-\dfrac{1}{4}
F^{\prime\mu\nu}F^{\prime}_{\mu\nu}-\dfrac{1}{4}\epsilon F^{\mu\nu}F^{\prime}_{\mu\nu}
\end{equation}

In Equation~\eqref{eq:lagrangian}, $F_{\mu\nu}$ is the electromagnetic field strength tensor defined in terms of the gradient of the potential as $\partial_{\mu}A_{\nu}-\partial_{\nu}A_{\mu}$. $F^{\prime}_{\mu\nu}$ corresponds to the field strength of the heavy photon and $\epsilon$ is the coupling. The third term of the Lagrangian is the kinetic mixing operator. The SM photon field can be re-defined as $A_{\mu}\rightarrow A_{\mu}+\epsilon A^{\prime}_{\mu}$ to remove the kinetic mixing operator. This generates a coupling to electric charge of order $\epsilon$ seen in the interaction between the heavy photon and SM as $\epsilon e A^{\prime}_{\mu}J^{\mu}_{EM}$ where $J^{\mu}_{EM}$ is the electromagnetic current.~\cite{toro} Particles that are charged only under the A$^{\prime}$ would not acquire this fractional charge and would remain undetectable in this model. \\
\indent The mass of the heavy photon is somewhat less constrained by theory. The MeV to GeV mass scale is interesting to explore because it has been generally overlooked in previous experimentation and is consistent with dark matter theories that attempt to explain several astrophysical phenomena. 

